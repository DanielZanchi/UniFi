\markboth{CAPITOLO 3. ESERCIZI A.A. 2017-18}{CAPITOLO 3. ESERCIZI A.A. 2017-18}
\begin{flushleft}
\textbf{Esercizio 3.1} \textit{Scrivere una function Matlab per la risoluzione di un sistema lineare con matrice dei coefficienti triangolare inferiore a diagonale unitaria. Inserire un esempio di utilizzo.}\\
\textbf{Soluzione: }
\lstinputlisting[language=Matlab]{Capitolo3/es3_1.m}
Il codice risolve sistemi lineari con matrice dei coefficenti triangolare inferiore a diagonale unitaria dove in input viene presa la matrice dei coefficienti \(A\) e il vettore dei termini noti \(b\) restituendo  vettore con le soluzioni.\\
Un esempio � il seguente:
\[
\begin{bmatrix}1 & 0 & 0 \\ 3 & 1 & 0\\ 4 & 1 & 2 \end{bmatrix} \bar{x} = \begin{bmatrix}1 \\ 2 \\ 3 \end{bmatrix}
\]
Il vettore delle soluzioni calcolato dalla funzione �: \(\begin{bmatrix}1 & -1 & 0 \end{bmatrix}\).\\
\bigskip
\textbf{Esercizio 3.2} \textit{Utilizzare l'Algoritmo 3.6 del libro per stabilire se le seguenti matrici sono sdp o no,}

\[
A_1 = \begin{pmatrix}1 & -1 & 2 & 2 \\ -1&5&-14&2\\ 2&-14&42&2\\2&2&2&65 \end{pmatrix},
A_2 = \begin{pmatrix}1 & -1 & 2 & 2 \\ -1&6&-17&3\\ 2&-17&48&-16\\2&3&-16&4 \end{pmatrix}
\]

\textbf{Soluzione: }Il seguente codice implementa l'algoritmo richiesto e si evince che la matrice $A_{1}$ � simmetrica e difinita positiva mentre la matrice $A_{2}$ non lo �:\\
\newpage
\lstinputlisting[language=Matlab]{Capitolo3/es3_2.m}
\bigskip
\textbf{Esercizio 3.3}
\textit{Scrivere una function Matlab che, avendo in ingresso un vettore b contenente i termini noti del sistema lineare \(Ax = b\) con \textit{A} sdp e l'output dell'Algoritmo 3.6 del libro (matrice \(A\) riscritta nella porzione triangolare inferiore con i fattori \(L\) e \(D\) della fattorizzazione \(LDL^T\) di \(A\)), ne calcoli efficientemente la soluzione.}\\
\textbf{Soluzione: }
\lstinputlisting[language=Matlab]{Capitolo3/es3_3.m}
La function \lstinline[language=Matlab]{sistema_triang_inf} � stata implementata nell'esercizio 1 di questo capitolo.\\
\bigskip
\textbf{Esercizio 3.4} \textit{Scrivere una function Matlab che, avendo in ingresso un vettore b contenente i termini noti del sistema lineare \(Ax = b\)  e l'output dell'Algoritmo 3.7 del libro (matrice \(A\) riscritta con la fattorizzazione LU con pivoting parziale e il vettore p delle permutazioni), ne calcoli efficientemente la soluzione.}\\
\textbf{Soluzione: }
Il seguente codice calcola quanto richiesto, le function \lstinline[language=Matlab]{sistema_triang_inf} e \lstinline[language=Matlab]{sistema_triang_sup} sono quelle degli Esercizi 1 e 3.\\
\lstinputlisting[language=Matlab]{Capitolo3/es3_4.m}
\bigskip
\textbf{Esercizio 3.5} \textit{Inserire alcuni esempi di utilizzo delle due function implementate per i punti 3 e 4, scegliendo per ciascuno di essi un vettore \(\hat{x}\) e ponendo \(b = A\hat{x}\). Riportare \(\hat{x}\) e la soluzione \(x\) da essi prodotta. Costruire anche una tabella in cui, per ogni esempio considerato, si riportano il numero di condizionamento di A in norma 2 (usare \textbf{cond} di Matlab) e le quantit� \(||r||/||b||\) e \(||x-\hat{x}||/||\hat{x}||\).}\\
\textbf{Soluzione: } Per dimostrare le function dell'esercizio 3 (risoluzione di sistemi $LDL^T$) e dell'esercizio 4 (scomposizione LU con pivoting parziale) verr� utilizzata la seguente matrice:\\
\[
A = \begin{pmatrix} 1 & -1 & 2 & 2 \\ -1 & 5 & -14 & 2\\ 2 & -14 & 42 & 2\\ 2 & 2 & 2 & 65 \end{pmatrix}
\]
\noindent I vettori scelti sono, rispettivamente per svolgere gli esercizi 3 e 4:

\[
\hat{x}_3 = \begin{bmatrix} 5.1211 \\ 3.4433 \\ 0.1257 \\ 2.1579 \end{bmatrix},
\hat{x}_4 = \begin{bmatrix} 1.3345 \\ 2.3232 \\ 3.1175 \\ 1.6658 \end{bmatrix}
\]

\noindent Ponendo \(b=A\hat{x}\) risulter� rispettivamente:
\\
\[
b_3 = \begin{bmatrix} 6.2450 \\ 14.6514 \\ -28.3688 \\ 157.6437 \end{bmatrix},
b_4 = \begin{bmatrix} 8.5779 \\ -30.0319 \\ 104.4108 \\ 121.8274 \end{bmatrix}
\]

\noindent Eseguendo le function create per la risoluzione del sistema viene generata la seguente soluzione:

\[
x_3 = \begin{bmatrix} 5.12110000000000 \\ 3.44330000000000 \\ 0.125699999999998 \\ 2.15790000000000 \end{bmatrix},
x_4 = \begin{bmatrix} 1.33450000000002 \\ 2.32320000000003 \\ 3.11750000000001 \\ 1.66580000000000 \end{bmatrix}
\]
\\
\noindent La tabella seguente mostra il condizionamento della matrice restituita dagli algoritmi di fattorizzazione ed i vari confronti di errori relativi sui dati di ingresso (\(||r||/||b||\)) e sul risultato (\(||x-\hat{x}||/||\hat{x}||\)):
\bigskip{}
\noindent\begin{tabular}{l*{20}{c}}
Fattorizzazione & \vline& \(\hat{x}\) & \vline& \(cond(A, 2)\) & \vline& \(||r||/||b||\) & \vline& \(||x-\hat{x}||/||\hat{x}||\) \\
\hline
\(LDL^T\)    & \vline& \(\hat{x}_3\) & \vline& \(3.6158 \times 10^3\) & \vline& \(4.4142 \times 10^{-17}\)	& \vline& \(8.3347 \times 10^{-16}\)   \\
\(LU\) pivot & \vline& \(\hat{x}_4\) & \vline& 319.1025 & \vline& \(9.0270 \times 10^{-17}\)	& \vline& \(8.4115 \times 10^{-15}\) 	 \\

\end{tabular}
\\

\noindent Il codice Matlab usato per realizzare i precedenti esempi � il seguente:\\
\lstinputlisting[language=Matlab]{Capitolo3/es3_5.m}
\bigskip{}
\textbf{Esercizio 3.6} \textit{Sia \(A = \begin{pmatrix} \epsilon & 1 \\ 1 & 1 \end{pmatrix}\) con $\epsilon=10^{-13}$.  Definire L triangolare inferiore a diagonale unitaria e U triangolare superiore in modo che il prodotto LU sia la fattorizzazione LU di A e, posto \(b=Ae\) con \(e=(1,1)^{T}\), confrontare l'accuratezza della soluzione che si ottiene usando il comando \(U \textbackslash (L \textbackslash b)\) (Gauss senza pivoting) e il comando \(A \textbackslash b\) (Gauss con pivoting).}\\
\textbf{Soluzione: }
Data la matrice A possiamo scrivere:\\
			\begin{center}
				$ A = L \times U =
				\begin{bmatrix}
				1 & 0 \\
				10^{13} & 1
				\end{bmatrix}
				\times
				\begin{bmatrix}
				10^{-13} & 1 \\
				0 & 1 - 10^{13}
				\end{bmatrix} $ \\
			\end{center}

			\lstinputlisting[language = matlab]{Capitolo3/es3_6.m}
			\noindent Calcoliamo \(b\):

			\[b = Ae = \begin{pmatrix} \epsilon & 1 \\ 1 & 1 \end{pmatrix} \begin{bmatrix} 1 \\ 1\end{bmatrix} = \begin{bmatrix} \approx 1 \\ 2 \end{bmatrix}\]
			\[U \textbackslash (L \textbackslash b) = \begin{bmatrix} 0.992 \\ 1\end{bmatrix}\]
			\[A \textbackslash b = \begin{bmatrix} 1 \\ 1\end{bmatrix}\]

			Dai calcoli richiesti segue che :\\
			\begin{center}
				$\varepsilon_{U \setminus (L \setminus b)} = 5.6517 \cdot 10^{-4}$ \hspace{10 pt} ed \hspace{10 pt} $ \varepsilon_{A \setminus b} = 2.2204 \cdot 10^{-16}$ \\
			\end{center}
			\noindent Si nota che il vettore \(b\) calcolato col metodo di Gauss senza pivoting abbia un'accuratezza minore rispetto alla sua versione calcolata con il metodo di Gauss con pivoting.
			\\
\bigskip
\textbf{Esercizio 3.7} \textit{Scrivere una function Matlab specifica per la risoluzione di un sistema lineare con matrice dei coefficienti \(A \in R^{n \times n}\) bidiagonale inferiore a diagonale unitaria di Toeplitz, specificabile con uno scalare \(\alpha\). Sperimentarne e commentarne le prestazioni (considerare il numero di condizionamento della matrice) nel caso in cui \(n=12\) e \(\alpha=100\) ponendo dapprima \(b=(1, 101, \ldots, 101)^T\) (soluzione esatta \(\hat{x}=(1,\ldots ,1)^T\)) e quindi \(b=0.1 * (1, 101, \ldots, 101)^T\) (soluzione esatta \(\hat{x}=(0.1,\ldots ,0.1)^T\)).}\\
\medskip
\textbf{Soluzione: } \noindent Ricordando che le matrici bidiagonali inferiori a diagonale unitaria di Toeplitz sono le matrici \(A \in R^{n \times n}\) del tipo \[\begin{pmatrix} 1 & 0 & 0 & \cdots & 0 \\ \alpha & 1 & 0 & \cdots & 0 \\ 0 & \alpha & 1 & \cdots & 0 \\ \vdots & \vdots & \vdots & \vdots & \vdots \\ 0 & \cdots & 0 & \alpha & 1\end{pmatrix}\]
\lstinputlisting[language=Matlab]{Capitolo3/es3_7.m}
Il seguente codice seguente applica le due function appena mostrate calcolando il condizionamento del problema e la soluzione nei due casi richiesti con $n = 12$ e $\alpha = 100$:\\
\lstinputlisting[language=Matlab]{Capitolo3/es3_7_esempio.m}
Il numero di conzionamento risulta essere infinito.
Di seguito vengono mostrati i risultati ottenuti nel primo caso e nel secondo caso relative a $x_{1}$ e $x_{2}$:
\[
x_1=\begin{bmatrix} 1 \\1\\1\\1\\1\\1\\1\\1\\1\\1\\1\\1\end{bmatrix}^T\qquad x_2=\begin{bmatrix}0.1000\\ 0.1000 \\ 0.1000 \\ 0.1000 \\ 0.1000 \\ 0.1000 \\ 0.1000 \\ 0.1014 \\ -0.0407 \\ 14.1702 \\ -1.4069 \times 10^3 \\ 1.4070 \times 10^5 \end{bmatrix}^T
\]\\
\bigskip
\textbf{Esercizio 3.8} \textit{Scrivere una function che, dato un sistema lineare sovradeterminato Ax = b, con \(A \in R^{m \times n}\), \(m > n\), \(rank(A)=n\) e \(b \in R^m\), preso come input b e l'output dell'Algoritmo 3.8 del libro (matrice A riscritta con la parte significativa dei vettori di Householder normalizzati con prima componente unitaria), ne calcoli efficientemente la soluzione nel senso dei minimi quadrati.}\\
\textbf{Soluzione: }
\lstinputlisting[language=Matlab]{Capitolo3/es3_8.m}
\bigskip
\textbf{Esercizio 3.9} \textit{Inserire due esempi diutilizzo della function implementata per il punto 8 e confrontare la soluzione ottenuta con quella fornita dal comando \(A \textbackslash b\).}\\
\textbf{Soluzione: }Per il primo esempio utilizzeremo:\\
\[
A_1 = \begin{pmatrix} 4 & 2 & 2 \\ 1 & 1& 2 \\ 2 & 3 & 4 \\ 2 & 1 & 1 \end{pmatrix} \quad b_1 = \begin{bmatrix} 4\\5\\4\\1 \end{bmatrix}
\]\\
La funzione restituisce:
\[
	x_{1} = \begin{bmatrix}1.533333333333334 \\ -5.99999999999999\\ 4.73333333333333 \end{bmatrix}^T \qquad e \qquad A_{1}\textbackslash b_{1} = \begin{bmatrix}1.53333333333333 \\ -6.00000000000000 \\ 4.73333333333333 \end{bmatrix}^T
\]\\
Per il secondo esempio invece:
\[
A_1 = \begin{pmatrix} 1 & 2 \\ 2 & 6 \\ 4 & 3 \end{pmatrix} \quad b_1 = \begin{bmatrix} 4\\5\\6 \end{bmatrix}
\]\\
La funzione restituisce:
\[
	x_{1} = \begin{bmatrix}1.15014164305949 \\ 0.532577903682720 \end{bmatrix}^T \qquad e \qquad A_{1}\textbackslash b_{1} = \begin{bmatrix}1.15014164305949 \\ 0.532577903682719 \end{bmatrix}^T
\]\\
In entrambi i casi $A\textbackslash b \approx x$\\
Il codice Matlab:
\lstinputlisting[language=Matlab]{Capitolo3/es3_9.m}
\bigskip
\textbf{Esercizio 3.10} \textit{Scrivere una function Matlab che realizza il metodo di Newton per un sistema nonlineare (prevedere un numero massimo di iterazioni e utilizzare il criteri di arresto basato sull'incremento in norma euclidea). Utilizzare la function costruita al punto 4 per la risoluzione del sistema lineare ad ogni iterazione.}\\
\textbf{Soluzione: }\lstinputlisting[language=Matlab]{Capitolo3/es3_10.m}
\bigskip
\textbf{Esercizio 3.11} \textit{Verificato che la funzione \(f(x_1,x_2) = x_1^2+x_2^3-x_1x_2\) ha un punto di minimo relativo in (1/12, 1/6), costruire una tabella in cui si riportano il numero di iterazioni eseguite, e la norma euclidea dell'ultimo incremento e quella dell'errore con cui viene approssimato il risultato esatto utilizzando la function sviluppata al punto precedente per valori delle tolleranze pari a \(10^{-t}\), con t = 3,6. Utilizzare (1/2, 1/2) come punto di innesco. Verificare che la norma dell'errore � molto pi� piccola di quella dell'incremento (come mai?)}\\
\textbf{Soluzione: }
\noindent Verifichiamo l'esistenza di un punto di minimo relativo in \((\frac{1}{12}, \frac{1}{6})\) considerando il sistema non lineare:
\[
F(\vec{x}) = \vec{0} \quad \text{con} \quad F = \begin{bmatrix}\frac{\partial}{\partial x_1}f(x_1,x_2) \\ \frac{\partial}{\partial x_2}f(x_1,x_2)  \end{bmatrix}
\]
\[
\frac{\partial}{\partial x_1}f(x_1,x_2) = 2x_1 -x_2 \quad \frac{\partial}{\partial x_2}f(x_1,x_2) = 3x_2^2 - x_1
\]
\[
\begin{cases}
2x_1 -x_2 = 0 \\
3y^2 - x_1 = 0
\end{cases}
\quad \text{ha come soluzioni} \quad \begin{bmatrix}0 & 0 \end{bmatrix} \quad \text{e} \quad \begin{bmatrix}\frac{1}{12} & \frac{1}{6}\end{bmatrix}
\]
\\
\noindent I punti trovati sono punti stazionari della funzione data.\\ Consideriamo la matrice Hessiana della funzione \(f(x,x_2)\), che coincide con la matrice Jacobiana della funzione \(F\):
\[
H =
\begin{bmatrix} f_{x_1x_1} & f_{x_1x_2} \\ f_{x_2x_1} & f_{x_2x_2} \end{bmatrix}
=
\begin{bmatrix} 2 & -1 \\ -1 & 6x_2 \end{bmatrix}
= J_F
\quad
\det(H) = 12x_2 - 1
\]

\noindent Il determinante della matrice Hessiana � positivo per \(x_2=\frac{1}{6}\), ed il primo elemento � positivo qui abbiamo un punto di minimo relativo in \((\frac{1}{12}, \frac{1}{6})\).
\\
\bigskip
\noindent I seguenti dati sono stati ottenuti considerando come soluzione esatta \(\hat{x} = [\frac{1}{12}, \frac{1}{6}]\) e come ultimo incremento la quantit� \(||x^{(i)} - x^{(i-1)}||\):
\\

\[
\begin{tabular}{l*{15}{c}}
 Tolleranza & \vline& iterazioni & \vline& \(||x^{(i)} - x^{(i-1)}||\) & \vline& \(||x - \hat{x}||\) \\
\hline
 \(10^{-3}\) & \vline& 5 &\vline& 2.8369 \(\times 10^{-4}\) & \vline& 4.3190 \(\times 10^{-7}\) \\
 \(10^{-4}\) & \vline& 6 &\vline& 4.3190 \(\times 10^{-7}\) & \vline& 1.0011 \(\times 10^{-12}\) \\
 \(10^{-5}\) & \vline& 6 &\vline& 4.3190 \(\times 10^{-7}\) & \vline& 1.0011 \(\times 10^{-12}\) \\
 \(10^{-6}\) & \vline& 6 &\vline& 4.3190 \(\times 10^{-7}\) & \vline& 1.0011 \(\times 10^{-12}\) \\
\end{tabular}
\] \\
La norma dell'ultimo incremento sara' molto minore rispetto alla norma dell'errore di approssimanzione del risultato. Avendo che il metodo di Newton ha ordine di convergenza pari a 2, che consente all'approssimazione del risultato di convergere velocemente verso il risultato esatto.
\lstinputlisting[language=Matlab]{Capitolo3/es3_11.m}

\end{flushleft}
