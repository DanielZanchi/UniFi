\markboth{CAPITOLO 5. ESERCIZI A.A. 2017-18}{CAPITOLO 5. ESERCIZI A.A. 2017-18}
\begin{flushleft}
\textbf{Esercizio 5.1} \textit{Scrivere una function Matlab che implementi la formula composita dei trapezi su \(n+1\) ascisse equidistanti nell'intervallo \([a, b]\) relativamente alla funzione implementata da \lstinline[language=Matlab]{Capitolo5/fun(x)}.\\La function deve essere del tipo \lstinline[language=Matlab]{Capitolo5/If = trapcomp(a, b, fun, tol)}.}\\
\textbf{Soluzione: }\lstinputlisting[language=Matlab]{Capitolo5/es5_1.m}
\bigskip
\textbf{Esercizio 5.2} \textit{Scrivere una function Matlab che implementi la formula composita di Simpson su \(2n+1\) ascisse equidistanti nell'intervallo \([a, b,]\) relativamente alla funzione implementata da \lstinline[language=Matlab]{Capitolo5/fun(x)}.\\La function deve essere del tipo \lstinline[language=Matlab]{Capitolo5/If = simpcomp(a, b, fun, tol)}.}\\
\textbf{Soluzione: }\lstinputlisting[language=Matlab]{Capitolo5/es5_2.m}
\bigskip
\textbf{Esercizio 5.3} \textit{Scrivere una function Matlab che implementi la formula composita dei trapezi adattativa nell'intervallo \([a, b]\) relativamente alla funzione implementata da \lstinline[language=Matlab]{Capitolo5/fun(x)} e con tolleranza \lstinline[language=Matlab]{Capitolo5/tol}.\\La function deve essere del tipo \lstinline[language=Matlab]{Capitolo5/If = trapad(a, b, fun, tol)}.}\\
\textbf{Soluzione: }\lstinputlisting[language=Matlab]{Capitolo5/es5_3.m}
\bigskip
\textbf{Esercizio 5.4} \textit{Scrivere una function Matlab che implementi la formula composita di Simpson adattativa nell'intervallo \([a, b]\) relativamente alla funzione implementata da \lstinline[language=Matlab]{Capitolo5/fun(x)} e con tolleranza \lstinline[language=Matlab]{Capitolo5/tol}.\\La function deve essere del tipo \lstinline[language=Matlab]{Capitolo5/If = simpad(a, b, fun, tol)}.}\\
\textbf{Soluzione: }\lstinputlisting[language=Matlab]{Capitolo5/es5_4.m}
\bigskip
\textbf{Esercizio 5.5} \textit{Calcolare quante valutazioni di funzione sono necessarie per ottenere una approssimazione di
\[I(f) = \int_0^1 \exp(-10^6 x) dx \]
che vale $10^{-6}$ in doppia precisione IEEE, con una tolleranza $10^{-9}$, utilizzando le functions dei precedenti esercizi. Argomentare quantitativamente la risposta.}\\
\textbf{Soluzione:} Le functions degli esercizi precedenti esercizi sono state leggermente modificate e sono stati calcolati i seguenti valori:\\
\[
\begin{tabular}{l*{15}{c}}
	Function & \vline & Valutazioni & \vline & Errore \\
	\hline
	Trapezi composita & \vline & 10000001 & \vline & $8.3319 \times 10^{-10}$ \\
	Simpson composita & \vline & 2000002 & \vline & $3.3715 \times 10^{-10}$ \\
	Trapezi adattiva & \vline & 77823 & \vline & $1.1252 \times 10^{-14}$\\
	Simpson adattiva & \vline & 1038 & \vline & $1.6469 \times 10^{-14}$\\
\end{tabular}
\]
Possiamo vedere che per la funzione presa in esame performano meglio le formule adattive in quanto individuano i nodi della partizione in base al comportamento locale della funzione, questo permette di minimizzare l'errore e di raggiungere la soglia di tolleranza prestabilita.\\
Viceversa le formule con ascisse equispaziate si sono rilevate inadeguate per questo tipo di funzione che presenta una rapida variazione di valore in una porzione dell'intervallo molto ristretta.\\
Il codice utilizzato per ottenere i risultati posti sopra:
\lstinputlisting[language=Matlab]{Capitolo5/es5_5.m}
\end{flushleft}
